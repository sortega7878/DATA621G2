\documentclass[]{article}
\usepackage{lmodern}
\usepackage{amssymb,amsmath}
\usepackage{ifxetex,ifluatex}
\usepackage{fixltx2e} % provides \textsubscript
\ifnum 0\ifxetex 1\fi\ifluatex 1\fi=0 % if pdftex
  \usepackage[T1]{fontenc}
  \usepackage[utf8]{inputenc}
\else % if luatex or xelatex
  \ifxetex
    \usepackage{mathspec}
  \else
    \usepackage{fontspec}
  \fi
  \defaultfontfeatures{Ligatures=TeX,Scale=MatchLowercase}
\fi
% use upquote if available, for straight quotes in verbatim environments
\IfFileExists{upquote.sty}{\usepackage{upquote}}{}
% use microtype if available
\IfFileExists{microtype.sty}{%
\usepackage{microtype}
\UseMicrotypeSet[protrusion]{basicmath} % disable protrusion for tt fonts
}{}
\usepackage[margin=1in]{geometry}
\usepackage{hyperref}
\hypersetup{unicode=true,
            pdftitle={HW2},
            pdfauthor={Chester Poon},
            pdfborder={0 0 0},
            breaklinks=true}
\urlstyle{same}  % don't use monospace font for urls
\usepackage{color}
\usepackage{fancyvrb}
\newcommand{\VerbBar}{|}
\newcommand{\VERB}{\Verb[commandchars=\\\{\}]}
\DefineVerbatimEnvironment{Highlighting}{Verbatim}{commandchars=\\\{\}}
% Add ',fontsize=\small' for more characters per line
\usepackage{framed}
\definecolor{shadecolor}{RGB}{248,248,248}
\newenvironment{Shaded}{\begin{snugshade}}{\end{snugshade}}
\newcommand{\KeywordTok}[1]{\textcolor[rgb]{0.13,0.29,0.53}{\textbf{#1}}}
\newcommand{\DataTypeTok}[1]{\textcolor[rgb]{0.13,0.29,0.53}{#1}}
\newcommand{\DecValTok}[1]{\textcolor[rgb]{0.00,0.00,0.81}{#1}}
\newcommand{\BaseNTok}[1]{\textcolor[rgb]{0.00,0.00,0.81}{#1}}
\newcommand{\FloatTok}[1]{\textcolor[rgb]{0.00,0.00,0.81}{#1}}
\newcommand{\ConstantTok}[1]{\textcolor[rgb]{0.00,0.00,0.00}{#1}}
\newcommand{\CharTok}[1]{\textcolor[rgb]{0.31,0.60,0.02}{#1}}
\newcommand{\SpecialCharTok}[1]{\textcolor[rgb]{0.00,0.00,0.00}{#1}}
\newcommand{\StringTok}[1]{\textcolor[rgb]{0.31,0.60,0.02}{#1}}
\newcommand{\VerbatimStringTok}[1]{\textcolor[rgb]{0.31,0.60,0.02}{#1}}
\newcommand{\SpecialStringTok}[1]{\textcolor[rgb]{0.31,0.60,0.02}{#1}}
\newcommand{\ImportTok}[1]{#1}
\newcommand{\CommentTok}[1]{\textcolor[rgb]{0.56,0.35,0.01}{\textit{#1}}}
\newcommand{\DocumentationTok}[1]{\textcolor[rgb]{0.56,0.35,0.01}{\textbf{\textit{#1}}}}
\newcommand{\AnnotationTok}[1]{\textcolor[rgb]{0.56,0.35,0.01}{\textbf{\textit{#1}}}}
\newcommand{\CommentVarTok}[1]{\textcolor[rgb]{0.56,0.35,0.01}{\textbf{\textit{#1}}}}
\newcommand{\OtherTok}[1]{\textcolor[rgb]{0.56,0.35,0.01}{#1}}
\newcommand{\FunctionTok}[1]{\textcolor[rgb]{0.00,0.00,0.00}{#1}}
\newcommand{\VariableTok}[1]{\textcolor[rgb]{0.00,0.00,0.00}{#1}}
\newcommand{\ControlFlowTok}[1]{\textcolor[rgb]{0.13,0.29,0.53}{\textbf{#1}}}
\newcommand{\OperatorTok}[1]{\textcolor[rgb]{0.81,0.36,0.00}{\textbf{#1}}}
\newcommand{\BuiltInTok}[1]{#1}
\newcommand{\ExtensionTok}[1]{#1}
\newcommand{\PreprocessorTok}[1]{\textcolor[rgb]{0.56,0.35,0.01}{\textit{#1}}}
\newcommand{\AttributeTok}[1]{\textcolor[rgb]{0.77,0.63,0.00}{#1}}
\newcommand{\RegionMarkerTok}[1]{#1}
\newcommand{\InformationTok}[1]{\textcolor[rgb]{0.56,0.35,0.01}{\textbf{\textit{#1}}}}
\newcommand{\WarningTok}[1]{\textcolor[rgb]{0.56,0.35,0.01}{\textbf{\textit{#1}}}}
\newcommand{\AlertTok}[1]{\textcolor[rgb]{0.94,0.16,0.16}{#1}}
\newcommand{\ErrorTok}[1]{\textcolor[rgb]{0.64,0.00,0.00}{\textbf{#1}}}
\newcommand{\NormalTok}[1]{#1}
\usepackage{longtable,booktabs}
\usepackage{graphicx,grffile}
\makeatletter
\def\maxwidth{\ifdim\Gin@nat@width>\linewidth\linewidth\else\Gin@nat@width\fi}
\def\maxheight{\ifdim\Gin@nat@height>\textheight\textheight\else\Gin@nat@height\fi}
\makeatother
% Scale images if necessary, so that they will not overflow the page
% margins by default, and it is still possible to overwrite the defaults
% using explicit options in \includegraphics[width, height, ...]{}
\setkeys{Gin}{width=\maxwidth,height=\maxheight,keepaspectratio}
\IfFileExists{parskip.sty}{%
\usepackage{parskip}
}{% else
\setlength{\parindent}{0pt}
\setlength{\parskip}{6pt plus 2pt minus 1pt}
}
\setlength{\emergencystretch}{3em}  % prevent overfull lines
\providecommand{\tightlist}{%
  \setlength{\itemsep}{0pt}\setlength{\parskip}{0pt}}
\setcounter{secnumdepth}{0}
% Redefines (sub)paragraphs to behave more like sections
\ifx\paragraph\undefined\else
\let\oldparagraph\paragraph
\renewcommand{\paragraph}[1]{\oldparagraph{#1}\mbox{}}
\fi
\ifx\subparagraph\undefined\else
\let\oldsubparagraph\subparagraph
\renewcommand{\subparagraph}[1]{\oldsubparagraph{#1}\mbox{}}
\fi

%%% Use protect on footnotes to avoid problems with footnotes in titles
\let\rmarkdownfootnote\footnote%
\def\footnote{\protect\rmarkdownfootnote}

%%% Change title format to be more compact
\usepackage{titling}

% Create subtitle command for use in maketitle
\newcommand{\subtitle}[1]{
  \posttitle{
    \begin{center}\large#1\end{center}
    }
}

\setlength{\droptitle}{-2em}

  \title{HW2}
    \pretitle{\vspace{\droptitle}\centering\huge}
  \posttitle{\par}
    \author{Chester Poon}
    \preauthor{\centering\large\emph}
  \postauthor{\par}
    \date{}
    \predate{}\postdate{}
  

\begin{document}
\maketitle

\paragraph{1) Download the classification output data set (attached in
Blackboard to the
assignment).}\label{download-the-classification-output-data-set-attached-in-blackboard-to-the-assignment.}

The dataset was downloaded and we load this data into a dataframe with
the below code:

\begin{Shaded}
\begin{Highlighting}[]
\NormalTok{df <-}\StringTok{ }\KeywordTok{read.csv}\NormalTok{(}\StringTok{'classification-output-data.csv'}\NormalTok{)}
\end{Highlighting}
\end{Shaded}

\paragraph{2) Use the table() function to get the raw confusion matrix
for this scored dataset. Make sure you understand the output. In
particular, do the rows represent the actual or predicted class? The
columns?}\label{use-the-table-function-to-get-the-raw-confusion-matrix-for-this-scored-dataset.-make-sure-you-understand-the-output.-in-particular-do-the-rows-represent-the-actual-or-predicted-class-the-columns}

The below code creates our confusion matrix:

\begin{Shaded}
\begin{Highlighting}[]
\KeywordTok{library}\NormalTok{(dplyr)}
\NormalTok{df1 <-}\StringTok{ }\NormalTok{df }\OperatorTok
\StringTok{  }\KeywordTok{select}\NormalTok{(scored.class,class)}
\NormalTok{cmatrix =}\StringTok{ }\KeywordTok{table}\NormalTok{(df1)}
\NormalTok{cmatrix}
\end{Highlighting}
\end{Shaded}

\begin{verbatim}
##             class
## scored.class   0   1
##            0 119  30
##            1   5  27
\end{verbatim}

The class columns are the actual/observed, while the scored.class rows
are the predicted.

\paragraph{3) Write a function that takes the data set as a dataframe,
with actual and predicted classifications identified, and returns the
accuracy of the
predictions.}\label{write-a-function-that-takes-the-data-set-as-a-dataframe-with-actual-and-predicted-classifications-identified-and-returns-the-accuracy-of-the-predictions.}

The below code coverts our matrix into a dataframe:

\begin{Shaded}
\begin{Highlighting}[]
\NormalTok{cmatrix <-}\StringTok{ }\KeywordTok{as.data.frame.matrix}\NormalTok{(cmatrix)}
\NormalTok{knitr}\OperatorTok{::}\KeywordTok{kable}\NormalTok{(cmatrix)}
\end{Highlighting}
\end{Shaded}

\begin{longtable}[]{@{}lrr@{}}
\toprule
& 0 & 1\tabularnewline
\midrule
\endhead
0 & 119 & 30\tabularnewline
1 & 5 & 27\tabularnewline
\bottomrule
\end{longtable}

And now we create our function that takes in the confusion matrix as a
dataframe and returns the accuracy of the predictions:

\begin{Shaded}
\begin{Highlighting}[]
\NormalTok{acc <-}\StringTok{ }\ControlFlowTok{function}\NormalTok{(x) \{}
\NormalTok{  (x[}\DecValTok{1}\NormalTok{,}\DecValTok{1}\NormalTok{] }\OperatorTok{+}\StringTok{ }\NormalTok{x[}\DecValTok{2}\NormalTok{,}\DecValTok{2}\NormalTok{])}\OperatorTok{/}\KeywordTok{sum}\NormalTok{(x)}
\NormalTok{\}}
\KeywordTok{acc}\NormalTok{(cmatrix)}
\end{Highlighting}
\end{Shaded}

\begin{verbatim}
## [1] 0.8066298
\end{verbatim}

\paragraph{4) Write a function that takes the data set as a dataframe,
with actual and predicted classifications identified, and returns the
classification error rate of the
predictions.}\label{write-a-function-that-takes-the-data-set-as-a-dataframe-with-actual-and-predicted-classifications-identified-and-returns-the-classification-error-rate-of-the-predictions.}

The function that returns a classification error rate is below:

\begin{Shaded}
\begin{Highlighting}[]
\NormalTok{class_err <-}\StringTok{ }\ControlFlowTok{function}\NormalTok{(x) \{}
\NormalTok{  (x[}\DecValTok{1}\NormalTok{,}\DecValTok{2}\NormalTok{] }\OperatorTok{+}\StringTok{ }\NormalTok{x[}\DecValTok{2}\NormalTok{,}\DecValTok{1}\NormalTok{])}\OperatorTok{/}\KeywordTok{sum}\NormalTok{(x)}
\NormalTok{\}}
\KeywordTok{class_err}\NormalTok{(cmatrix)}
\end{Highlighting}
\end{Shaded}

\begin{verbatim}
## [1] 0.1933702
\end{verbatim}

\paragraph{Verify that you get an accuracy and an error rate that sums
to
one.}\label{verify-that-you-get-an-accuracy-and-an-error-rate-that-sums-to-one.}

When we add the accuracy with the error rate we get 1:

\begin{Shaded}
\begin{Highlighting}[]
\KeywordTok{class_err}\NormalTok{(cmatrix) }\OperatorTok{+}\StringTok{ }\KeywordTok{acc}\NormalTok{(cmatrix)}
\end{Highlighting}
\end{Shaded}

\begin{verbatim}
## [1] 1
\end{verbatim}

\paragraph{5) Write a function that takes the data set as a dataframe,
with actual and predicted classifications identified, and returns the
precision of the
predictions.}\label{write-a-function-that-takes-the-data-set-as-a-dataframe-with-actual-and-predicted-classifications-identified-and-returns-the-precision-of-the-predictions.}

The function that returns the precision:

\begin{Shaded}
\begin{Highlighting}[]
\NormalTok{prec <-}\StringTok{ }\ControlFlowTok{function}\NormalTok{(x) \{}
\NormalTok{  x[}\DecValTok{1}\NormalTok{,}\DecValTok{1}\NormalTok{]}\OperatorTok{/}\NormalTok{(x[}\DecValTok{1}\NormalTok{,}\DecValTok{1}\NormalTok{] }\OperatorTok{+}\StringTok{ }\NormalTok{x[}\DecValTok{1}\NormalTok{,}\DecValTok{2}\NormalTok{])}
\NormalTok{\}}
\KeywordTok{prec}\NormalTok{(cmatrix)}
\end{Highlighting}
\end{Shaded}

\begin{verbatim}
## [1] 0.7986577
\end{verbatim}

\paragraph{6) Write a function that takes the data set as a dataframe,
with actual and predicted classifications identified, and returns the
sensitivity of the predictions. Sensitivity is also known as
recall.}\label{write-a-function-that-takes-the-data-set-as-a-dataframe-with-actual-and-predicted-classifications-identified-and-returns-the-sensitivity-of-the-predictions.-sensitivity-is-also-known-as-recall.}

Sensitivity:

\begin{Shaded}
\begin{Highlighting}[]
\NormalTok{sensitivity <-}\StringTok{ }\ControlFlowTok{function}\NormalTok{(x) \{}
\NormalTok{  x[}\DecValTok{1}\NormalTok{,}\DecValTok{1}\NormalTok{]}\OperatorTok{/}\NormalTok{(x[}\DecValTok{1}\NormalTok{,}\DecValTok{1}\NormalTok{] }\OperatorTok{+}\StringTok{ }\NormalTok{x[}\DecValTok{2}\NormalTok{,}\DecValTok{1}\NormalTok{])}
\NormalTok{\}}
\KeywordTok{sensitivity}\NormalTok{(cmatrix)}
\end{Highlighting}
\end{Shaded}

\begin{verbatim}
## [1] 0.9596774
\end{verbatim}

\paragraph{7) Write a function that takes the data set as a dataframe,
with actual and predicted classifications identified, and returns the
specificity of the
predictions.}\label{write-a-function-that-takes-the-data-set-as-a-dataframe-with-actual-and-predicted-classifications-identified-and-returns-the-specificity-of-the-predictions.}

Specificity:

\begin{Shaded}
\begin{Highlighting}[]
\NormalTok{specificity <-}\StringTok{ }\ControlFlowTok{function}\NormalTok{(x) \{}
\NormalTok{  x[}\DecValTok{2}\NormalTok{,}\DecValTok{2}\NormalTok{]}\OperatorTok{/}\NormalTok{(x[}\DecValTok{2}\NormalTok{,}\DecValTok{2}\NormalTok{] }\OperatorTok{+}\StringTok{ }\NormalTok{x[}\DecValTok{1}\NormalTok{,}\DecValTok{2}\NormalTok{])}
\NormalTok{\}}
\KeywordTok{specificity}\NormalTok{(cmatrix)}
\end{Highlighting}
\end{Shaded}

\begin{verbatim}
## [1] 0.4736842
\end{verbatim}

\paragraph{8) Write a function that takes the data set as a dataframe,
with actual and predicted classifications identified, and returns the F1
score of the
predictions.}\label{write-a-function-that-takes-the-data-set-as-a-dataframe-with-actual-and-predicted-classifications-identified-and-returns-the-f1-score-of-the-predictions.}

We can actually use some of our previously created functions to create
this function.

F1-Score:

\begin{Shaded}
\begin{Highlighting}[]
\NormalTok{f1 <-}\StringTok{ }\ControlFlowTok{function}\NormalTok{(x) \{}
\NormalTok{  (}\DecValTok{2}\OperatorTok{*}\KeywordTok{prec}\NormalTok{(x)}\OperatorTok{*}\KeywordTok{sensitivity}\NormalTok{(x))}\OperatorTok{/}\NormalTok{(}\KeywordTok{prec}\NormalTok{(x) }\OperatorTok{+}\StringTok{ }\KeywordTok{sensitivity}\NormalTok{(x))}
\NormalTok{\}}
\KeywordTok{f1}\NormalTok{(cmatrix)}
\end{Highlighting}
\end{Shaded}

\begin{verbatim}
## [1] 0.8717949
\end{verbatim}

\paragraph{9) Before we move on, let's consider a question that was
asked: What are the bounds on the F1 score? Show that the F1 score will
always be between 0 and 1. (Hint: If 0 \textless{} 𝑎 \textless{} 1 and 0
\textless{} 𝑏 \textless{} 1 then 𝑎𝑏 \textless{}
𝑎.)}\label{before-we-move-on-lets-consider-a-question-that-was-asked-what-are-the-bounds-on-the-f1-score-show-that-the-f1-score-will-always-be-between-0-and-1.-hint-if-0--1-and-0--1-then--.}

Because both precision and sensitivity will always be between 0 and 1,
if you use the upper and lower bounds of precision and sensivity, you
will always get a number between 0 and 1 from the formula for F1 score.

\paragraph{10) Write a function that generates an ROC curve from a data
set with a true classification column (class in our example) and a
probability column (scored.probability in our example). Your function
should return a list that includes the plot of the ROC curve and a
vector that contains the calculated area under the curve (AUC). Note
that I recommend using a sequence of thresholds ranging from 0 to 1 at
0.01
intervals.}\label{write-a-function-that-generates-an-roc-curve-from-a-data-set-with-a-true-classification-column-class-in-our-example-and-a-probability-column-scored.probability-in-our-example.-your-function-should-return-a-list-that-includes-the-plot-of-the-roc-curve-and-a-vector-that-contains-the-calculated-area-under-the-curve-auc.-note-that-i-recommend-using-a-sequence-of-thresholds-ranging-from-0-to-1-at-0.01-intervals.}

\begin{Shaded}
\begin{Highlighting}[]
\NormalTok{roc <-}\StringTok{ }\ControlFlowTok{function}\NormalTok{(actual, scores)\{}
\NormalTok{  actual <-}\StringTok{ }\NormalTok{actual[}\KeywordTok{order}\NormalTok{(scores, }\DataTypeTok{decreasing=}\OtherTok{TRUE}\NormalTok{)]}
  \KeywordTok{plot}\NormalTok{(}\KeywordTok{cumsum}\NormalTok{(}\OperatorTok{!}\NormalTok{actual)}\OperatorTok{/}\KeywordTok{sum}\NormalTok{(}\OperatorTok{!}\NormalTok{actual),}
       \KeywordTok{cumsum}\NormalTok{(actual)}\OperatorTok{/}\KeywordTok{sum}\NormalTok{(actual), }
       \DataTypeTok{type=}\StringTok{'l'}\NormalTok{,}
       \DataTypeTok{xlab =} \StringTok{"False Positive Rate"}\NormalTok{,}
       \DataTypeTok{ylab =} \StringTok{"True Positive Rate"}\NormalTok{)}
\NormalTok{  f <-}\StringTok{ }\KeywordTok{approxfun}\NormalTok{(}\KeywordTok{cumsum}\NormalTok{(}\OperatorTok{!}\NormalTok{actual)}\OperatorTok{/}\KeywordTok{sum}\NormalTok{(}\OperatorTok{!}\NormalTok{actual),}
                 \KeywordTok{cumsum}\NormalTok{(actual)}\OperatorTok{/}\KeywordTok{sum}\NormalTok{(actual))}
\NormalTok{  C <-}\StringTok{ }\KeywordTok{integrate}\NormalTok{(f, }
                 \KeywordTok{min}\NormalTok{(}\KeywordTok{cumsum}\NormalTok{(actual)}\OperatorTok{/}\KeywordTok{sum}\NormalTok{(actual)), }
                 \KeywordTok{max}\NormalTok{(}\KeywordTok{cumsum}\NormalTok{(actual)}\OperatorTok{/}\KeywordTok{sum}\NormalTok{(actual)))}\OperatorTok{$}\NormalTok{value}
  \KeywordTok{paste}\NormalTok{(}\StringTok{"AUC = "}\NormalTok{,C)}
\NormalTok{\}}

\KeywordTok{roc}\NormalTok{(df}\OperatorTok{$}\NormalTok{class,df}\OperatorTok{$}\NormalTok{scored.probability)}
\end{Highlighting}
\end{Shaded}

\includegraphics{hw2_files/figure-latex/unnamed-chunk-11-1.pdf}

\begin{verbatim}
## [1] "AUC =  0.846480454012587"
\end{verbatim}

\paragraph{11) Use your created R functions and the provided
classification output data set to produce all of the classification
metrics discussed
above.}\label{use-your-created-r-functions-and-the-provided-classification-output-data-set-to-produce-all-of-the-classification-metrics-discussed-above.}

\begin{Shaded}
\begin{Highlighting}[]
\NormalTok{metric <-}\StringTok{ }\KeywordTok{c}\NormalTok{(}\StringTok{"Accuracy"}\NormalTok{,}\StringTok{"Classification Error"}\NormalTok{,}\StringTok{"Precision"}\NormalTok{,}\StringTok{"Sensitivity"}\NormalTok{,}\StringTok{"Specificity"}\NormalTok{,}\StringTok{"F1-Score"}\NormalTok{)}

\NormalTok{values <-}\StringTok{ }\KeywordTok{c}\NormalTok{(}\KeywordTok{acc}\NormalTok{(cmatrix),}
            \KeywordTok{class_err}\NormalTok{(cmatrix),}
            \KeywordTok{prec}\NormalTok{(cmatrix),}
            \KeywordTok{sensitivity}\NormalTok{(cmatrix),}
            \KeywordTok{specificity}\NormalTok{(cmatrix),}
            \KeywordTok{f1}\NormalTok{(cmatrix))}

\NormalTok{metric_df <-}\StringTok{ }\KeywordTok{data.frame}\NormalTok{(metric,values)}
\KeywordTok{names}\NormalTok{(metric_df) <-}\StringTok{ }\KeywordTok{c}\NormalTok{(}\StringTok{"Metric"}\NormalTok{,}\StringTok{"Values"}\NormalTok{)}

\NormalTok{knitr}\OperatorTok{::}\KeywordTok{kable}\NormalTok{(metric_df)}
\end{Highlighting}
\end{Shaded}

\begin{longtable}[]{@{}lr@{}}
\toprule
Metric & Values\tabularnewline
\midrule
\endhead
Accuracy & 0.8066298\tabularnewline
Classification Error & 0.1933702\tabularnewline
Precision & 0.7986577\tabularnewline
Sensitivity & 0.9596774\tabularnewline
Specificity & 0.4736842\tabularnewline
F1-Score & 0.8717949\tabularnewline
\bottomrule
\end{longtable}

\paragraph{12) Investigate the caret package. In particular, consider
the functions confusionMatrix, sensitivity, and specificity. Apply the
functions to the data set. How do the results compare with your own
functions?}\label{investigate-the-caret-package.-in-particular-consider-the-functions-confusionmatrix-sensitivity-and-specificity.-apply-the-functions-to-the-data-set.-how-do-the-results-compare-with-your-own-functions}

\begin{Shaded}
\begin{Highlighting}[]
\KeywordTok{library}\NormalTok{(caret)}
\KeywordTok{confusionMatrix}\NormalTok{(}\KeywordTok{factor}\NormalTok{(df1}\OperatorTok{$}\NormalTok{scored.class),}\KeywordTok{factor}\NormalTok{(df1}\OperatorTok{$}\NormalTok{class))}
\end{Highlighting}
\end{Shaded}

\begin{verbatim}
## Confusion Matrix and Statistics
## 
##           Reference
## Prediction   0   1
##          0 119  30
##          1   5  27
##                                           
##                Accuracy : 0.8066          
##                  95% CI : (0.7415, 0.8615)
##     No Information Rate : 0.6851          
##     P-Value [Acc > NIR] : 0.0001712       
##                                           
##                   Kappa : 0.4916          
##  Mcnemar's Test P-Value : 4.976e-05       
##                                           
##             Sensitivity : 0.9597          
##             Specificity : 0.4737          
##          Pos Pred Value : 0.7987          
##          Neg Pred Value : 0.8438          
##              Prevalence : 0.6851          
##          Detection Rate : 0.6575          
##    Detection Prevalence : 0.8232          
##       Balanced Accuracy : 0.7167          
##                                           
##        'Positive' Class : 0               
## 
\end{verbatim}

The values above from the \texttt{confusionMatrix} return identical
values as the manually created functions from earlier.

The below two functions from the \texttt{caret} package returns the same
sensitivity and specificity outputs as the \texttt{confusionMatrix}
function above.

\begin{Shaded}
\begin{Highlighting}[]
\NormalTok{caret}\OperatorTok{::}\KeywordTok{sensitivity}\NormalTok{(}\KeywordTok{factor}\NormalTok{(df1}\OperatorTok{$}\NormalTok{scored.class),}\KeywordTok{factor}\NormalTok{(df1}\OperatorTok{$}\NormalTok{class))}
\end{Highlighting}
\end{Shaded}

\begin{verbatim}
## [1] 0.9596774
\end{verbatim}

\begin{Shaded}
\begin{Highlighting}[]
\NormalTok{caret}\OperatorTok{::}\KeywordTok{specificity}\NormalTok{(}\KeywordTok{factor}\NormalTok{(df1}\OperatorTok{$}\NormalTok{scored.class),}\KeywordTok{factor}\NormalTok{(df1}\OperatorTok{$}\NormalTok{class))}
\end{Highlighting}
\end{Shaded}

\begin{verbatim}
## [1] 0.4736842
\end{verbatim}

\paragraph{13) Investigate the pROC package. Use it to generate an ROC
curve for the data set. How do the results compare with your own
functions?}\label{investigate-the-proc-package.-use-it-to-generate-an-roc-curve-for-the-data-set.-how-do-the-results-compare-with-your-own-functions}

\begin{Shaded}
\begin{Highlighting}[]
\KeywordTok{library}\NormalTok{(pROC)}
\KeywordTok{par}\NormalTok{(}\DataTypeTok{pty=}\StringTok{"s"}\NormalTok{)}
\NormalTok{pROC}\OperatorTok{::}\KeywordTok{roc}\NormalTok{(df}\OperatorTok{$}\NormalTok{class,df}\OperatorTok{$}\NormalTok{scored.probability,}
          \DataTypeTok{plot=}\OtherTok{TRUE}\NormalTok{, }
          \DataTypeTok{legacy.axes=}\OtherTok{TRUE}\NormalTok{,}
          \DataTypeTok{print.auc=}\OtherTok{TRUE}\NormalTok{)}
\end{Highlighting}
\end{Shaded}

\includegraphics{hw2_files/figure-latex/unnamed-chunk-15-1.pdf}

\begin{verbatim}
## 
## Call:
## roc.default(response = df$class, predictor = df$scored.probability,     plot = TRUE, legacy.axes = TRUE, print.auc = TRUE)
## 
## Data: df$scored.probability in 124 controls (df$class 0) < 57 cases (df$class 1).
## Area under the curve: 0.8503
\end{verbatim}

The ROC curves appear identical and the AUC values are approximately the
same.


\end{document}
